% Options for packages loaded elsewhere
\PassOptionsToPackage{unicode}{hyperref}
\PassOptionsToPackage{hyphens}{url}
%
\documentclass[
]{article}
\usepackage{amsmath,amssymb}
\usepackage{iftex}
\ifPDFTeX
  \usepackage[T1]{fontenc}
  \usepackage[utf8]{inputenc}
  \usepackage{textcomp} % provide euro and other symbols
\else % if luatex or xetex
  \usepackage{unicode-math} % this also loads fontspec
  \defaultfontfeatures{Scale=MatchLowercase}
  \defaultfontfeatures[\rmfamily]{Ligatures=TeX,Scale=1}
\fi
\usepackage{lmodern}
\ifPDFTeX\else
  % xetex/luatex font selection
\fi
% Use upquote if available, for straight quotes in verbatim environments
\IfFileExists{upquote.sty}{\usepackage{upquote}}{}
\IfFileExists{microtype.sty}{% use microtype if available
  \usepackage[]{microtype}
  \UseMicrotypeSet[protrusion]{basicmath} % disable protrusion for tt fonts
}{}
\makeatletter
\@ifundefined{KOMAClassName}{% if non-KOMA class
  \IfFileExists{parskip.sty}{%
    \usepackage{parskip}
  }{% else
    \setlength{\parindent}{0pt}
    \setlength{\parskip}{6pt plus 2pt minus 1pt}}
}{% if KOMA class
  \KOMAoptions{parskip=half}}
\makeatother
\usepackage{xcolor}
\usepackage[margin=1.5cm]{geometry}
\usepackage{color}
\usepackage{fancyvrb}
\newcommand{\VerbBar}{|}
\newcommand{\VERB}{\Verb[commandchars=\\\{\}]}
\DefineVerbatimEnvironment{Highlighting}{Verbatim}{commandchars=\\\{\}}
% Add ',fontsize=\small' for more characters per line
\usepackage{framed}
\definecolor{shadecolor}{RGB}{248,248,248}
\newenvironment{Shaded}{\begin{snugshade}}{\end{snugshade}}
\newcommand{\AlertTok}[1]{\textcolor[rgb]{0.94,0.16,0.16}{#1}}
\newcommand{\AnnotationTok}[1]{\textcolor[rgb]{0.56,0.35,0.01}{\textbf{\textit{#1}}}}
\newcommand{\AttributeTok}[1]{\textcolor[rgb]{0.13,0.29,0.53}{#1}}
\newcommand{\BaseNTok}[1]{\textcolor[rgb]{0.00,0.00,0.81}{#1}}
\newcommand{\BuiltInTok}[1]{#1}
\newcommand{\CharTok}[1]{\textcolor[rgb]{0.31,0.60,0.02}{#1}}
\newcommand{\CommentTok}[1]{\textcolor[rgb]{0.56,0.35,0.01}{\textit{#1}}}
\newcommand{\CommentVarTok}[1]{\textcolor[rgb]{0.56,0.35,0.01}{\textbf{\textit{#1}}}}
\newcommand{\ConstantTok}[1]{\textcolor[rgb]{0.56,0.35,0.01}{#1}}
\newcommand{\ControlFlowTok}[1]{\textcolor[rgb]{0.13,0.29,0.53}{\textbf{#1}}}
\newcommand{\DataTypeTok}[1]{\textcolor[rgb]{0.13,0.29,0.53}{#1}}
\newcommand{\DecValTok}[1]{\textcolor[rgb]{0.00,0.00,0.81}{#1}}
\newcommand{\DocumentationTok}[1]{\textcolor[rgb]{0.56,0.35,0.01}{\textbf{\textit{#1}}}}
\newcommand{\ErrorTok}[1]{\textcolor[rgb]{0.64,0.00,0.00}{\textbf{#1}}}
\newcommand{\ExtensionTok}[1]{#1}
\newcommand{\FloatTok}[1]{\textcolor[rgb]{0.00,0.00,0.81}{#1}}
\newcommand{\FunctionTok}[1]{\textcolor[rgb]{0.13,0.29,0.53}{\textbf{#1}}}
\newcommand{\ImportTok}[1]{#1}
\newcommand{\InformationTok}[1]{\textcolor[rgb]{0.56,0.35,0.01}{\textbf{\textit{#1}}}}
\newcommand{\KeywordTok}[1]{\textcolor[rgb]{0.13,0.29,0.53}{\textbf{#1}}}
\newcommand{\NormalTok}[1]{#1}
\newcommand{\OperatorTok}[1]{\textcolor[rgb]{0.81,0.36,0.00}{\textbf{#1}}}
\newcommand{\OtherTok}[1]{\textcolor[rgb]{0.56,0.35,0.01}{#1}}
\newcommand{\PreprocessorTok}[1]{\textcolor[rgb]{0.56,0.35,0.01}{\textit{#1}}}
\newcommand{\RegionMarkerTok}[1]{#1}
\newcommand{\SpecialCharTok}[1]{\textcolor[rgb]{0.81,0.36,0.00}{\textbf{#1}}}
\newcommand{\SpecialStringTok}[1]{\textcolor[rgb]{0.31,0.60,0.02}{#1}}
\newcommand{\StringTok}[1]{\textcolor[rgb]{0.31,0.60,0.02}{#1}}
\newcommand{\VariableTok}[1]{\textcolor[rgb]{0.00,0.00,0.00}{#1}}
\newcommand{\VerbatimStringTok}[1]{\textcolor[rgb]{0.31,0.60,0.02}{#1}}
\newcommand{\WarningTok}[1]{\textcolor[rgb]{0.56,0.35,0.01}{\textbf{\textit{#1}}}}
\usepackage{graphicx}
\makeatletter
\def\maxwidth{\ifdim\Gin@nat@width>\linewidth\linewidth\else\Gin@nat@width\fi}
\def\maxheight{\ifdim\Gin@nat@height>\textheight\textheight\else\Gin@nat@height\fi}
\makeatother
% Scale images if necessary, so that they will not overflow the page
% margins by default, and it is still possible to overwrite the defaults
% using explicit options in \includegraphics[width, height, ...]{}
\setkeys{Gin}{width=\maxwidth,height=\maxheight,keepaspectratio}
% Set default figure placement to htbp
\makeatletter
\def\fps@figure{htbp}
\makeatother
\setlength{\emergencystretch}{3em} % prevent overfull lines
\providecommand{\tightlist}{%
  \setlength{\itemsep}{0pt}\setlength{\parskip}{0pt}}
\setcounter{secnumdepth}{-\maxdimen} % remove section numbering
\ifLuaTeX
  \usepackage{selnolig}  % disable illegal ligatures
\fi
\usepackage{bookmark}
\IfFileExists{xurl.sty}{\usepackage{xurl}}{} % add URL line breaks if available
\urlstyle{same}
\hypersetup{
  pdftitle={Lista 1 - Estatística Computacional},
  pdfauthor={Prof.~Juliano Bortolini},
  hidelinks,
  pdfcreator={LaTeX via pandoc}}

\title{Lista 1 - Estatística Computacional}
\author{Prof.~Juliano Bortolini}
\date{}

\begin{document}
\maketitle

\begin{enumerate}
\def\labelenumi{\arabic{enumi}.}
\item
  Usando o \emph{software} R, resolva as expressões numéricas:

  \begin{enumerate}
  \def\labelenumii{\alph{enumii}.}
  \tightlist
  \item
    \(\frac{1}{2} + \frac{1}{3}\)
  \item
    \(\frac{4}{3} - \left(- \frac{2}{3} \right) + \frac{5}{2}\)
  \item
    \(4 - \left( - \frac{2}{3} + \frac{1}{4} - \frac{1}{8} \right) + \frac{1}{5}\)
  \item
    \(\left( - \frac{1}{2}\right)^2 \left( - \frac{1}{3}\right) + \left( \frac{1}{4}\right)^{-1}\)
  \item
    \(\frac{\sqrt{0,25 \cdot 0,04} + \sqrt[3]{0,01 \cdot 0,0001}}{\sqrt{0,32 \div 0,02} - \sqrt{0,5^4}}\)
  \item
    \(\frac{\left(\frac{1}{3}\right)^{-2}}{\left(\frac{2}{3}\right)^{-2} - \left(\frac{3}{2}\right)^{-2}}\)
  \item
    \(\frac{5^{-1}}{10^{-1} - 2^{-1}}\)
  \item
    \(\left(-\frac{1}{2} \right)^2 \left(-\frac{1}{3} \right) + \left(-\frac{2}{3} \right)^{-1} \div \left(-\frac{1}{3} \right)^{-2}\)
  \item
    \(\left( \frac{1}{3} + \frac{2}{5} - \frac{7}{2}\right) - \left( -\frac{1}{2} + \frac{1}{3} + \frac{5}{4}\right)\)
  \item
    \(\frac{1}{2} - \left[ \frac{5}{3} - \left( \frac{1}{4} + \frac{1}{3}\right) \right]\)
  \item
    \(\left[\left(\frac{1}{2} + \sqrt{2}\right)^{-2} + \frac{\pi^2}{\frac{\sqrt{3}}{1 + 0,333...}}\right] - \log_{10}(16 + 7,5^2)\)
  \end{enumerate}
\item
  Execute os comandos \texttt{sqrt(16)}, \texttt{16\^{}0.5} e
  \texttt{4\^{}3}.
\item
  Execute os comandos \texttt{log10(1000)}, \texttt{log(1000)},
  \texttt{exp(log\ (1000))}. Então, tente o comando \texttt{log2(64)}.
  Quais as diferenças entre as funções executadas?
\item
  Experimente o comando \texttt{?log}. Leia as primeiras linhas. O texto
  corresponde às suas observações do exercício anterior?
\item
  Execute os comandos \texttt{pi}, \texttt{round(pi)},
  \texttt{round(pi,\ digits\ =\ 4)} e \texttt{trunc(pi)}.
\item
  As funções seno e cosseno são implementadas em \texttt{sin} e
  \texttt{cos}. Calcule \texttt{sin(pi)}, \texttt{cos(pi)},
  \texttt{sin(pi/2)}, \texttt{cos(pi/2)}.
\item
  Execute os comandos a seguir e observe os resultados. Note que as
  operações são realizadas elemento a elementos.
\end{enumerate}

\begin{Shaded}
\begin{Highlighting}[]
\NormalTok{  x }\OtherTok{\textless{}{-}} \FunctionTok{c}\NormalTok{(}\DecValTok{3}\NormalTok{,}\DecValTok{6}\NormalTok{,}\DecValTok{8}\NormalTok{)}
\NormalTok{  x}
\NormalTok{  x}\SpecialCharTok{/}\DecValTok{2}
\NormalTok{  x}\SpecialCharTok{\^{}}\DecValTok{2}
  \FunctionTok{sqrt}\NormalTok{(x)}
\NormalTok{  x[}\DecValTok{2}\NormalTok{]}
\NormalTok{  x[}\FunctionTok{c}\NormalTok{(}\DecValTok{1}\NormalTok{,}\DecValTok{3}\NormalTok{)]}
\NormalTok{  x[}\SpecialCharTok{{-}}\DecValTok{3}\NormalTok{]}
\NormalTok{  y }\OtherTok{\textless{}{-}} \FunctionTok{c}\NormalTok{(}\DecValTok{2}\NormalTok{,}\DecValTok{5}\NormalTok{,}\DecValTok{1}\NormalTok{)}
\NormalTok{  y}
\NormalTok{  x}\SpecialCharTok{{-}}\NormalTok{y}
\NormalTok{  x}\SpecialCharTok{*}\NormalTok{y}
\NormalTok{  x[y }\SpecialCharTok{\textgreater{}} \FloatTok{1.5}\NormalTok{]}
\NormalTok{  y[x }\SpecialCharTok{==} \DecValTok{6}\NormalTok{]}
  \DecValTok{4}\SpecialCharTok{:}\DecValTok{10}
  \FunctionTok{seq}\NormalTok{(}\DecValTok{2}\NormalTok{,}\DecValTok{3}\NormalTok{, }\AttributeTok{by =} \FloatTok{0.1}\NormalTok{)}
  \FunctionTok{rep}\NormalTok{(x, }\AttributeTok{each =} \DecValTok{4}\NormalTok{)}
\end{Highlighting}
\end{Shaded}

\begin{enumerate}
\def\labelenumi{\arabic{enumi}.}
\setcounter{enumi}{7}
\item
  Suponha que registramos a altura e o peso para quatro pessoas: alturas
  em cm são 180, 165, 160, 193; pesos em kg são 87, 58, 65, 100. Faça
  dois vetores, altura e peso, com os dados. O índice de massa corporal
  (IMC) é definido como \[
   \frac{peso \ (\text{em kg})}{altura^2 \ (\text{em m}) }
  \] Faça um vetor com os valores do IMC para as quatro pessoas, e um
  vetor com o logaritmo natural para os valores do IMC. Finalmente, faça
  um vetor com os pesos para essas pessoas que têm um IMC maior que 25.
\item
  Mostrar os comandos que podem ser usados para criar as sequências:
\end{enumerate}

\begin{verbatim}
## [1] 1 2 3
\end{verbatim}

\begin{verbatim}
##  [1] 10 11 12 13 14 15 16 17 18 19 20
\end{verbatim}

\begin{verbatim}
## [1] -3 -2 -1  0  1  2  3
\end{verbatim}

\begin{verbatim}
## [1] -8 -7 -6 -5
\end{verbatim}

\begin{verbatim}
##  [1] 2.4 2.9 3.4 3.9 4.4 4.9 5.4 5.9 6.4 6.9 7.4
\end{verbatim}

\begin{verbatim}
##  [1]  0.0  1.1  2.2  3.3  4.4  5.5  6.6  7.7  8.8  9.9 11.0
\end{verbatim}

\begin{verbatim}
##  [1]  0  2  4  6  8 10 12 14 16 18 20
\end{verbatim}

\begin{verbatim}
##  [1]  1  3  5  7  9 11 13 15 17 19
\end{verbatim}

\begin{verbatim}
##  [1] 1 1 2 2 3 3 4 4 5 5
\end{verbatim}

\begin{verbatim}
##  [1] 1 2 3 4 5 1 2 3 4 5
\end{verbatim}

\begin{verbatim}
##  [1] 1 2 2 3 3 3 4 4 4 4 5 5 5 5 5
\end{verbatim}

\begin{verbatim}
##  [1] 1 1 1 1 1 2 2 2 2 3 3 3 4 4 5
\end{verbatim}

\begin{verbatim}
##  [1]  0.0  2.5  2.5  5.0  5.0  5.0  7.5  7.5  7.5  7.5 10.0 10.0 10.0 10.0 10.0
\end{verbatim}

\begin{verbatim}
## [1] "a" "b" "c" "d" "e"
\end{verbatim}

\begin{verbatim}
## [1] "c" "f" "i" "l"
\end{verbatim}

\begin{verbatim}
## [1] "H" "G" "F" "E" "D" "C" "B"
\end{verbatim}

\begin{verbatim}
## [1] "A" "C" "E" "G" "I" "K"
\end{verbatim}

\begin{verbatim}
## [1] "T1" "T2" "T3" "T1" "T2" "T3" "T1" "T2" "T3"
\end{verbatim}

\begin{verbatim}
## [1] "T1" "T1" "T2" "T2" "T3" "T3" "T3" "T3"
\end{verbatim}

\begin{enumerate}
\def\labelenumi{\arabic{enumi}.}
\setcounter{enumi}{9}
\item
  Suponha que você marcou 10 tempos que gasta para executar uma tarefa.
  Os tempos em minutos foram: 18, 14, 14, 15, 14, 34, 16, 17, 21, 26.
  Armazene esses valores em um objeto com o nome \texttt{tempo}. Usando
  funções do R obtenha o tempo máximo, mínimo e o tempo médio.
\item
  (continuando o exercício anterior) O valor 34 foi um erro, ele na
  verdade é 15. Sem digitar tudo novamente, e usando colchetes
  \texttt{{[}{]}}, altere o valor e calcule novamente o tempo médio.
\item
  Você consegue prever o resultado dos comandos abaixo? Caso não
  consiga, execute os comandos e veja o resultado:
\end{enumerate}

\begin{Shaded}
\begin{Highlighting}[]
\NormalTok{x}\OtherTok{\textless{}{-}}\FunctionTok{c}\NormalTok{(}\DecValTok{1}\NormalTok{,}\DecValTok{3}\NormalTok{,}\DecValTok{5}\NormalTok{,}\DecValTok{7}\NormalTok{,}\DecValTok{9}\NormalTok{)}
\NormalTok{y}\OtherTok{\textless{}{-}}\FunctionTok{c}\NormalTok{(}\DecValTok{2}\NormalTok{,}\DecValTok{3}\NormalTok{,}\DecValTok{5}\NormalTok{,}\DecValTok{7}\NormalTok{,}\DecValTok{11}\NormalTok{,}\DecValTok{13}\NormalTok{)}
\NormalTok{x}\SpecialCharTok{+}\DecValTok{1}
\NormalTok{y}\SpecialCharTok{*}\DecValTok{2}
\FunctionTok{length}\NormalTok{(x)}
\FunctionTok{length}\NormalTok{(y)}
\NormalTok{x }\SpecialCharTok{+}\NormalTok{ y}
\NormalTok{y[}\DecValTok{3}\NormalTok{]}
\NormalTok{y[}\SpecialCharTok{{-}}\DecValTok{3}\NormalTok{]}
\end{Highlighting}
\end{Shaded}

\begin{enumerate}
\def\labelenumi{\arabic{enumi}.}
\setcounter{enumi}{12}
\item
  Calcule a velocidade média de um objeto que percorreu 150 km em 2.5
  horas. \textbf{Sugestão:} lembre-se da fórmula do cálculo da
  velocidade média.
\item
  Calcule \(|2^3-3^2|\)
\item
  Suponha que você deseje jogar na mega-sena, mas não sabe quais números
  jogar, use a função sample do R para escolher seis números para você.
  Lembre que a mega-sena tem valores de 1 a 60.
\item
  Crie uma sequencia de dados de 1 a 30 apenas com números impares. Use
  a função \texttt{seq()}.
\item
  Utilizando apenas as funções \texttt{c()}, \texttt{seq()} e indexação
  de vetores \texttt{{[}{]}}, crie os seguintes objetos:

  \begin{enumerate}
  \def\labelenumii{\alph{enumii}.}
  \tightlist
  \item
    Uma sequência de quinze valores em intervalos regulares, indo de 0 a
    100, nomeada sq1.
  \item
    Um objeto, denominado sq2, que contenha todos os elementos de sq1,
    exceto o quinto e décimo valores.
  \item
    Um vetor sq3 contendo apenas as posições ímpares do objeto sq1.
  \item
    Uma sequência igual a sq1 substituindo, apenas os valores nas
    posições pares, pelo número relativo à sua posição. Denomine esse
    objeto de sq4.
  \end{enumerate}
\item
  Abaixo estão listadas as distâncias por estradas entre quatro cidades
  da Europa, em quilômetros:

  \begin{itemize}
  \tightlist
  \item
    Atenas a Madri: 3949
  \item
    Atenas a Paris: 3000
  \item
    Atenas a Estocolmo: 3927
  \item
    Madri a Paris: 1273
  \item
    Madri a Estocolomo: 3188
  \item
    Paris a Estocolmo: 1827
  \end{itemize}
\end{enumerate}

Crie um objeto da classe matrix denominado dist.cid com os valores
acima. Nesta matriz, a diagonal principal deve conter zeros e o
``triângulo'' acima da diagonal principal deve conter as mesmas
informações do ``triângulo'' abaixo da diagonal principal. Para
facilitar o uso desse objeto, o nome das linhas e das colunas deve ser o
nome das cidades. Você consegue pensar em duas formas diferentes de
criar a matriz com nomes nas linhas e colunas? \textbf{Sugestão:}
utilize as funções lower.tri,upper.tri e diag.

\begin{enumerate}
\def\labelenumi{\arabic{enumi}.}
\setcounter{enumi}{18}
\item
  Faça um programa que receba quatro números, calcule e mostre a soma
  desses números.
\item
  Faça um programa que receba três notas, calcule e mostre a média
  aritmética.
\item
  Faça um programa que receba três notas e seus respectivos pesos,
  calcule e mostre a média ponderada.
\item
  Faça um programa que receba o salário de um funcionário, calcule e
  mostre o novo salário, sabendo-se que este sofreu um aumento de
  \(25\%\).
\item
  Faça um programa que receba o salário de um funcionário e o percentual
  de aumento, calcule e mostre o valor do aumento e o novo salário.
\item
  Faça um programa que receba o salário base de um funcionário, calcule
  e mostre o salário a receber, sabendo-se que o funcionário tem
  gratificação de \(128\) reais sobre o salário base e paga os impostos
  de \(14\%\) e \(25/%
  \), referentes a previdência e de renda. Note que não há incidência de
  imposto na gratificação e a ordem de desconto dos impostos é primeiro
  o da previdência e posteriormente o de renda.
\item
  Faça um programa que receba o valor de um depósito, o valor da taxa de
  juros compostos e o tempo de aplicação. Calcule e mostre o valor do
  rendimento e o valor total depois do rendimento.
\item
  Faça um programa que calcule e mostre a área de um triângulo.
\item
  Faça um programa que calcule e mostre a área de um círculo.
\item
  Faça um programa que receba um número positivo e maior que zero,
  calcule e mostre:

  \begin{enumerate}
  \def\labelenumii{\alph{enumii}.}
  \tightlist
  \item
    número digitado ao quadrado;
  \item
    o número digitado ao cubo;
  \item
    a raiz quadrado do número digitado;
  \item
    a raiz cúbica do número digitado.
  \end{enumerate}
\item
  Faça um programa que receba dois números e mostre o maior.
\item
  Darth Vader comprou um saco de ração com peso em quilos. Ele possui
  dois gatos, para os quais fornece a quantidade de ração em gramas. A
  quantidade diária de ração fornecida para cada gato é sempre a mesma.
  Faça um programa que receba o peso do saco de ração e a quantidade de
  ração fornecida para cada gato, calcule e mostre quanto restará no
  saco após cinco dias.
\item
  Cada degrau de uma escada tem \(x\) de altura. Faça um programa que
  receba essa altura e a altura que o usuário deseja alcançar subindo a
  escada, calcule e mostre quantos degraus ele deverá subir para atingir
  seu objetivo, sem se preocupar com a altura do usuário. Todas as
  medidas fornecidas devem estar em metros.
\item
  Faça um programa que receba um conjunto de valores e mostre o mínimo,
  mediana, máximo, média, amplitude, variância, desvio padrão e a
  quantidade de elementos do conjunto de dados.
\item
  Faça um programa que receba um número inteiro e verifique se é par ou
  ímpar.
\item
  Faça um programa para resolver equações do segundo grau (
  \(ax^2 + bx + c = 0\) ).
\end{enumerate}

35.Dados três valores \(x, y, z\), verifique se eles podem ser os
comprimentos dos lados de um triângulo e, se forem, verifique se é um
triângulo equilátero, isósceles ou escaleno. Considere que:

\begin{itemize}
\tightlist
\item
  O comprimento de cada lado de um triângulo é menor que a soma dos
  outros dois lados.
\item
  Chama-se equilátero o triângulo que tem os três lados iguais
\item
  Denomina-se isósceles o triângulo que tem o comprimento de dois lados
  iguais.
\item
  Recebe o nome de escaleno o triângulo que tem os três lados
  diferentes.
\end{itemize}

\begin{enumerate}
\def\labelenumi{\arabic{enumi}.}
\setcounter{enumi}{35}
\tightlist
\item
  Implementar uma função no R para realizar o teste \emph{t} de
  \emph{Student} para duas amostras independentes e normais. Considerar
  os casos de variâncias heterogêneas e homogêneas. Utilizar uma
  estrutura condicional para aplicar o teste apropriado, caso as
  variâncias sejam heterogêneas ou homogêneas.
\end{enumerate}

\textbf{Sugestões:}

\begin{itemize}
\tightlist
\item
  utilizar a função \texttt{bartlett.test()} (teste de Bartlett) para
  homogeneidade de variâncias.
\item
  na estrutura condicional referente a homogeneidade de variâncias,
  considere o valor-p do teste de Bartlett.
\end{itemize}

\begin{enumerate}
\def\labelenumi{\arabic{enumi}.}
\setcounter{enumi}{36}
\item
  Construa uma função que apresenta \(n\) termos da sequência de
  Fibonacci.
\item
  Considerando a função do exercício anterior, construa uma nova função
  que apresenta o \texttt{n-ésimo} termo da sequência de Fibonacci.
\item
  Considerando a função que obtém uma estimativa do intervalo de
  confiança para a média de uma população normal de variância
  desconhecida:
\end{enumerate}

\begin{Shaded}
\begin{Highlighting}[]
\NormalTok{IC.t }\OtherTok{\textless{}{-}} \ControlFlowTok{function}\NormalTok{(x,}\AttributeTok{alpha =} \FloatTok{0.05}\NormalTok{)\{}
\NormalTok{  n }\OtherTok{\textless{}{-}} \FunctionTok{length}\NormalTok{(x)}
\NormalTok{  IC }\OtherTok{\textless{}{-}} \FunctionTok{c}\NormalTok{(}\FunctionTok{mean}\NormalTok{(x) }\SpecialCharTok{{-}} \FunctionTok{qt}\NormalTok{(}\DecValTok{1} \SpecialCharTok{{-}}\NormalTok{ alpha}\SpecialCharTok{/}\DecValTok{2}\NormalTok{, n}\DecValTok{{-}1}\NormalTok{) }\SpecialCharTok{*} \FunctionTok{sd}\NormalTok{(x)}\SpecialCharTok{/}\FunctionTok{sqrt}\NormalTok{(n),}
\FunctionTok{mean}\NormalTok{(x) }\SpecialCharTok{+} \FunctionTok{qt}\NormalTok{(}\DecValTok{1} \SpecialCharTok{{-}}\NormalTok{ alpha}\SpecialCharTok{/}\DecValTok{2}\NormalTok{, n}\DecValTok{{-}1}\NormalTok{) }\SpecialCharTok{*} \FunctionTok{sd}\NormalTok{(x)}\SpecialCharTok{/}\FunctionTok{sqrt}\NormalTok{(n))}
  \FunctionTok{return}\NormalTok{(}\AttributeTok{IC =}\NormalTok{ IC)}
\NormalTok{\}}
\end{Highlighting}
\end{Shaded}

Construa uma função que obtém \(m\) amostras aleatórias de tamanho \(n\)
da distribuição normal de média \(\mu\) e desvio padrão \(\sigma\)
(função \texttt{rnorm(n,\ mean,\ sd)}), obtém estimativas de intervalos
de confiança \(1 - \alpha\) para cada amostra e, considerando o valor
paramétrico \(\mu\), calcula a proporção de intervalos que contém o
valor do parâmetro \(\mu\).

\begin{enumerate}
\def\labelenumi{\arabic{enumi}.}
\setcounter{enumi}{39}
\item
  Descrever a função \texttt{apply()} (já implementada no R), e
  apresentar exemplo\textbf{s} de sua utilização e refaça o exercício
  anterior usando a função \texttt{apply}.
\item
  Construa uma função que calcula o traço de uma matriz \(\mathbf{A}\)
  de dimensões \(n \times n\), sendo o traço a soma dos elementos da
  diagonal principal, ou seja,
  \(tr(\mathbf{A}) = \sum_{i = 1}^n a_{ii}\). Acrescente um aviso se a
  matriz \(\mathbf{A}\) não for quadrada.
\item
  Desenvolva uma função em linguagem R que calcula a média geométrica de
  um vetor \(\mathbf{x} = (x_1,\cdots,x_n)\) e retorna uma mensagem de
  erro se algum elemento de \(\mathbf{x}\) for negativo ou nulo. A média
  geométrica de \(\mathbf{x} = (x_1,\cdots,x_n)\) é calculada por \[
  mg = \left(\prod_{i = 1}^n x_i \right)^{1/n}.
  \] \textbf{Sugestão:} considere as funções \texttt{any()} e
  \texttt{stop()}.
\end{enumerate}

43 Desenvolva uma função em linguagem R que calcula a média harmônica
ponderada de um vetor \(\mathbf{x} = (x_1,\cdots,x_n)\) com pesos
\(\mathbf{w} = (w_1,\cdots,w_n)\) e retorna uma mensagem de erro se
algum elemento de \(\mathbf{x}\) ou \(\mathbf{w}\) for negativo ou nulo.
A média harmônica ponderada de \(\mathbf{x} = (x_1,\cdots,x_n)\) com
pesos \(\mathbf{w} = (w_1,\cdots,w_n)\) é calculada por \[
mh = \frac{\sum_{i = 1}^n w_i}{\sum_{i = 1}^n \frac{w_i}{x_i} }.
\] \textbf{Sugestão:} considere as funções \texttt{any()} e
\texttt{stop()}.

\begin{enumerate}
\def\labelenumi{\arabic{enumi}.}
\setcounter{enumi}{43}
\tightlist
\item
  A função que se segue baseia-se no crivo de Eratóstenes, o mais antigo
  método sistemático para listar os números primos até um determinado
  valor \(n\) . A ideia é a seguinte: iniciar com um vetor de números de
  \(2\) a \(n\). Começando com \(2\), eliminar todos os múltiplos de
  \(2\), que são maiores do que \(2\). Em seguida, passar para o próximo
  número restante no vetor, neste caso, \(3\). Agora, remova todos os
  múltiplos de \(3\), que são maiores do que \(3\). Continue com todos
  os valores restantes do vetor. O valor \(4\) foi removido na primeira
  iteração, deixando \(5\) como o próximo elemento a ser considerado
  após o \(3\); todos os múltiplos de \(5\) será removido no passo
  seguinte e assim por diante.
\end{enumerate}

\begin{quote}
Fonte: W. John Braun, Duncan J. Murdoch. (2008) A first course in
statistical programming with R. Cambridge University Press.
\end{quote}

Considerando a função em R a seguir, descreva detalhadamente sobre o que
é a função e comente cada linha de instrução do código.

\begin{Shaded}
\begin{Highlighting}[]
\NormalTok{Eratostenes }\OtherTok{\textless{}{-}} \ControlFlowTok{function}\NormalTok{(n) \{ }\CommentTok{\# ?}
\ControlFlowTok{if}\NormalTok{ (n }\SpecialCharTok{\%\%} \DecValTok{1} \SpecialCharTok{!=} \DecValTok{0} \SpecialCharTok{|}\NormalTok{ n }\SpecialCharTok{\textless{}} \DecValTok{2}\NormalTok{) }\FunctionTok{stop}\NormalTok{(}\StringTok{"O valor de n deve ser inteiro e \textgreater{}= 2."}\NormalTok{) }\CommentTok{\# ?}
\NormalTok{crivo }\OtherTok{\textless{}{-}} \FunctionTok{seq}\NormalTok{(}\DecValTok{2}\NormalTok{, n) }\CommentTok{\# ?}
\NormalTok{primos }\OtherTok{\textless{}{-}} \ConstantTok{NULL} \CommentTok{\# ?}
\ControlFlowTok{for}\NormalTok{ (i }\ControlFlowTok{in} \FunctionTok{seq}\NormalTok{(}\DecValTok{2}\NormalTok{, n)) \{ }\CommentTok{\# ?}
\ControlFlowTok{if}\NormalTok{ (}\FunctionTok{any}\NormalTok{(crivo }\SpecialCharTok{==}\NormalTok{ i)) \{ }\CommentTok{\# ?}
\NormalTok{primos }\OtherTok{\textless{}{-}} \FunctionTok{c}\NormalTok{(primos, i) }\CommentTok{\# ?}
\NormalTok{crivo }\OtherTok{\textless{}{-}} \FunctionTok{c}\NormalTok{(crivo[(crivo }\SpecialCharTok{\%\%}\NormalTok{ i) }\SpecialCharTok{!=} \DecValTok{0}\NormalTok{], i) }\CommentTok{\# ?}
\NormalTok{\} }\CommentTok{\# ?}
\NormalTok{\} }\CommentTok{\# ?}
\FunctionTok{return}\NormalTok{(primos) }\CommentTok{\# ?}
\NormalTok{\} }\CommentTok{\# ?}

\FunctionTok{Eratostenes}\NormalTok{(}\DecValTok{0}\NormalTok{) }\CommentTok{\# ?}
\NormalTok{Error }\ControlFlowTok{in} \FunctionTok{Eratostenes}\NormalTok{(}\DecValTok{0}\NormalTok{) }\SpecialCharTok{:}\NormalTok{ O valor de n deve ser inteiro e }\SpecialCharTok{\textgreater{}=} \FloatTok{2.} \CommentTok{\# ?}

\FunctionTok{Eratostenes}\NormalTok{(}\DecValTok{12}\NormalTok{) }\CommentTok{\# ?}
\NormalTok{[}\DecValTok{1}\NormalTok{]  }\DecValTok{2}  \DecValTok{3}  \DecValTok{5}  \DecValTok{7} \DecValTok{11} \CommentTok{\# ?}
\end{Highlighting}
\end{Shaded}


\end{document}
